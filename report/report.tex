\documentclass{article}
\usepackage[utf8]{inputenc}

\title{Course 02285: AI and MAS}
\author{Team BotBot: Simon S. Linneberg S152408, Søren Pilgård S160521}
\date{2/6 2016}

\begin{document}

\maketitle

\section{Abstract}
Focus on Solving fast.\\
\section{Introduction}
Going from heiracical plan with merging to pure lowlevel.\\

\section{Theory}
- APSP\\
- online/offline\\
- Different types of tasks somewhat similar to FSM\\
- Definition of rooms.\\
- reservations.\\
\subsection{Heuristics}
- storage calculation.\\
- goal count and distance.\\
- An agent can un-solve a goal after placing it.\\
- no recalculation of APSP.\\
- goal priority.\\
- Excluding multiple agents in the same corridor.\\ 
\subsection{Multiagent communication}
- Take the top actions, works well with APSP if no conflicts.\\
- This is not as fast as solving individual agents with no other moving (as in presentation)\\
- plan merging could be better, there is no priority.\\
- Nops only when merging.\\
\section{Experiments}
\subsection{Results}
- Did well on the speed tracks (MA, only one level was solved)\\
- Not very many problems where solved, especially MA\\
\section{Discussion}
- Multiagent communication does not work well,\\
- Better way of controlling the different heuristics\\
- Better use of roads.\\
- Putting the agent on the right side of a box(according to next goal) in a corridor.\\
- Use of APSP means that we will always only use the shortest path. (use of wayoints)\\
- TODO: check that a path exists.\\
\section{Conclusion}
\section{References}
\end{document}

