\def\year{2016}
%File: formatting-instruction.tex
\documentclass[letterpaper]{article}

% Required Packages
\usepackage{aaai}
\usepackage{times}
\usepackage{helvet}
\usepackage{courier}
\usepackage[utf8]{inputenc}
\frenchspacing
\setlength{\pdfpagewidth}{8.5in}
\setlength{\pdfpageheight}{11in}

% Section numbers.
\setcounter{secnumdepth}{2}

\nocopyright


\title{Course 02285: AI and MAS - Team botbot}
\author{Simon Skjerning Linneberg\\ S152408 \And Søren Pilgård\\ S160521}
\date{2/6 2016}

\begin{document}

\maketitle

\begin{abstract}
Our solution to the AIMAS programming project, botbot, is an offline planner using A*
with an inadmissible heuristic. The heuristic is a weighted sum of different heuristics,
including the agent-box-goal distance, a storage estimate of unused blocks, and
the number of unfulfilled goals.\\

In multiagent levels all agents are aware of each other.\\

We focused on solving as many levels as fast as possible, and did not mind if we
got a longer solution if it was faster or could solve more levels. For a single
agent our solution is decent and could solve 5/14 single agent competition levels,
but the multiagent planer did not work very well, and could only solve one level,
although it got the fastest solution to that level.
\end{abstract}

\section{Introduction}
Initially we considered creating a Hierarchical Task Network (HTN). The highlevel
planner would decide what agents moved which box to what goal. The lowlevel planner
would then either move a box to a goal, or away to a storage area.

A problem of this strategy of turning our solution into a set of discrete
tasks was that not all problems can be solved by two or more such tasks in
sequence. We were able to create some fairly small levels that completely broke the
approach by having the algorithm trying to divide problem in various ways, while
a simple A* solution by the lowlevel planner looking at just the surrounding area had no problems.

Further there were some implementation problems, for example a task to move a box to a goal
would not tell us where the agent would be after the goal had been completed, which
could be important in a corridor where the box would block the agent. Another detail
is that each  highlevel action may be solved by a different number of lowlevel actions.

Instead we tried to flip the approach, letting the lowlevel planner create a
solution, guided by a set of current tasks. We assigned each goal a box of the
same letter based on the distance, and added a task to a list of tasks, so that
each agent can choose a fitting task when it is free. This solves the problems
above, and though it is not as 'smart' as doing an actual highlevel plan we
expect it to be as efficient in practice.\\

When an agent completes a task, the task is removed, and if an agent move a box away from
a goal, a new task is created. The tasks are thus stored together with the state.
We also add 'storage' task if an agent moves into a corridor and a box (which is not
part of the agents current task) is in the way (See the heuristic section).

\section{Methods}
First we describe the preprocessing done to facilitate fast heuristics, and then
what heuristics we use.

\subsection{Preprocessing}
We split the map into rooms and corridors. A corridor is a cell that has at most
two adjacent\footnote{To the north, south, east or west. Not north-west and other corners.} cells that are not walls, and is not a corner (if the cell is part of
a 2x2 square of cells that are not walls it is a corner). This is what we would
intuitively consider corridors and rooms. We create a map where each cell gives
the index to the corridor/room, and associate up to two connected rooms with each corridor,
and any number of corridors connected with rooms.

With this definition the cell of a T-intersection or cross is considered a room.\\

We calculate the storage value for each cell. A cell on a 'critical' path has storage
value 0. A critical path is the shortest path from a goal to its box, as calculated
by the initial tasks.

A cell in a corridor, and any cell in a room connected to a corridor has storage
value 1, and any other cell (interior of a room) has storage value 2.\\

The All Pairs Shortest Path (APSP) map is calculated once in the beginning, and never changed (see the discussion
part for alternatives). We do not consider boxes, goals or agents when calculating
this.\\

We try to establish a simple heuristic for which goals must be solved first.
This is the goal priority. The basic idea is that goals deep in coridors,
especially dead ends most likely has to be solved before goals at the start of
the corridor. To determine how ``deep'' into a corridor a goal is a flood-fill
like algorithm is used. The algorithm uses a queue of unfinished positions and a
map of ``colors'' for each position, the color or priority is an integer.
The color/priority under a robot is defined as 0, we then flood the color out,
when the color passes a goal it increases by one and thus colors the goal with
the new priority and flows on with this new value. A position is always given
the minimum priority, this way alternative routes can be taken without
increasing the priority. The end result is a basic map of how many goals has to
be crossed before getting to the goal from the closest robot.

The heuristic is fairly effective in many use cases as goals at the end of
corridors has to be solved first, unfortunately it can be fooled if a useless
starts behind it, thus negating the priority, or it can be fooled if the goals
has to be solved by backing in to them thus solving the lowest priority first.

\subsection{Heuristics}
The final weighted heuristic is:
\begin{verbatim}
w1 * totalDistance +
w2 * taskDistance +
w3 * storageTaskCount +
w4 * goalCount +
w5 * storagePunishment +
w6 * sameRoad +
w7 * modifier +
w8 * misconfiguration +
w9 * boxInCorridorPunishmet
\end{verbatim}
These are all described below.

We could not make misconfiguration and boxInCorridorPunishmet
work properly, so they where disabled in the final build.

\subsubsection{totalDistance}
If the agent is moving a box to a goal, this is the distance from the agent
to the box plus the distance from the box to its goal. If the agent is storing
a box, this is the distance from the agent to the box plus the distance from the
box to a nearby storage cell (see below).

This heuristic prioritize states where the agent is close to the box it is moving,
and where the box is close to its goal.

Since we have calculated APSP, this can be calculated effectively (not considering other boxes/agents)\\

\subsubsection{taskDistance}
taskDistance is the distance between each box and its goal summed over each task
that has not been assigned to an agent. This makes states more attractive, if
a box is moved closer to its goal, even if it is not the active task of an agent.

This should not dominate the totalDistance.

\subsubsection{goalCount}
The number of goal that does not have a (valid) box on top of them. This makes it
unattractive to move a box that already fulfills a goal, and was added when we found
out that agents would simply remove boxes that it had just moved to its goal.

\subsubsection{storageTaskCount}
The number of storage goal that is active, or needs to be choosen by an agent.
This is weighted negative (see modifier).

\subsubsection{goalCount}
This is the number of goal that does not have a (valid) box ontop. They are only
counted if the priority of the goal is equal to or higher than the maximum priority
goal that has not been completed.

This heuristic was introduced, since not considering the number of completed goals
would make the heuristic higher for a state where a task had just been completed,
since the task is replaced by a new task that would typically be much further away,
thus increasing the above heuristics. This might be considered a hack. The goalCount
heuristic must be weighted so high that it is better to choose the new task, than to
check the previous states, where that particular task had not been completed.

The reason that only the high priority goals are counted, is to encourage moving
boxes on low priority goals. The reason that they are low priority is because that
they may be in the way of other goals, so they must be able to be moved.

\subsubsection{storagePunishment}
Adds the storage level for each box that is not part of an active task, and
punishes based on the storage level of its position (see storage level).

This encourages agents to move boxes to free storage area.

\subsubsection{sameRoad}
If multiple agents are in the same corridor, this is punished, to refrain multiple
agents from blocking up a corridor. This could be improved, so that multiple agents
can go in the same direction, but we did not have time for that.

\subsubsection{modifier}
If a storage task is completed, it will be replaced by another task, which most likely
have a higher heuristic cost. Therefore a state where the storage task has not been completed
yet, will be preferred over the state where it has been completed, which is obviously bad
for A*. This was also a problem when completing a normal task, which could be countered
by punishing the number of unfinished tasks a lot.

This cannot be done in the case of storage tasks, since we do not have a maximum number of
storage tasks like in the case of ordinary tasks. If we add a storage task there will be
a task more than before, and the heuristic cost of completing the storage task will make that
state unattractive. Therefore storageTaskCount it weighted negative. This way having a storage
task will be more attractive that not. It must be even more attractive to finish the storage
task than to have it active. This is solved by adding the modifier count to the state.

Each time a storage task is completed, the modifier is incremented. Since the weight is
negative, it is attractive to have stored a box. It should not be more attractive than
completing a goal though.

\subsubsection{misconfiguration}
This is an estimate of how much corridors leading into a room is blocked.

\subsubsection{boxInCorridorPunishmet}
The number of agents that are pushing/pulling a box into a corridor when that box
is not part of the agents current task. This is to avoid blocking corridors with boxes.

\subsection{Multiagent communication}
Though our initial plan was to focus on the multiagent part, we realized that
a good single agent planner was needed to make multiagent planning work,
and since we are a small group we did not have time to improve this part.\\

Since we use the APSP-heuristic, it should be easy to solve a level if
there are no conflicts. Picking the action that leads
the agent closest to completition of its task, can be done for each agent,
and APSP would tell us what that action is. When there are no conflicts,
the individual actions can easily be merged.\\

In practice conflicts must be handled. This is done by letting each agent
genereate all possible actions. Then we pick the top action for each agent,
and try to merge them. The actions are added to the joint action one by one
sorted by the agents number. If an action conflicts with a previously added
action it is replaced with a noop.\\

Both the new state and the old state is put back on the queue of state that have
not been explored. If the old state is picked again, we pick the top two actions
for each agent, and generate all combinations of those actions. The resulting states
are put in the queue if they have not been visited before. We repeat this until
all combinations of actions the state have been checked, or we have found a solution.\\

The reasoning behind this is that if there are a low number of conflicts APSP
then the top agent actions would be valid, and we would rarely have to go back
to a prevous state. Finding the combinations as needed should improve memory usage.

In practice we doubt that this has any real effect (see Discussion).\\

In the presentations we saw how time effective it was to only let one agent move at
the same time. If the time taken really was our only priority we should probably
have done that aswell.

\subsection{Different tasks}
We have a task different tasks to put a box on a goal, and to move a box to
a storage area. This is used if a box blocks a corridor. While we do not explicitly 
model a FSM, we still implicit have a state for storage and a state for putting
boxes on goals. The agents transitions between these states when they enter
a corridor with a box, or when they complete a goal. This idea is then somewhat
similar to the FSM directed planning form \cite{JO}, although more subtle.

\section{Related work}
Our problem is somewhat similar to the sokoban game, which is well covered by
literature. Since agents in our problem can pull boxes back, and around corners,
we cannot reach a state that cannot be undone, and (assuming that all boxes are
reachable in the beginning) we cannot put a box where it cannot be reached. Two
things that must be considered in sokoban.

In \cite{sokobanMA} an overview of techniques used for solving sokoban is given.
We will now review the sections that are most interesting for our problem.\\

Instead of hashing the whole state (like we do) they suggest using transposition
tables where two states are considered the same if the boxes are at the same positions,
and the reachability area of the player is the same. This will make many states
the same, thus decreasing the size of the search space. This is appealing in
the sokoban problem, since the reachability area may permanently change. This
or a similar technique could perhaps be used to simplify the multiagent problem.\\

They also mention the use of 'intertia' to encourage pushing the same box.
We do not use inertia the same way, but our tasks has a similar effect by
always considering the same box. The inertia is a bit more flexible, since
it can change, while ours only changes if we create a storage goal.\\

Another technique in \cite{sokobanMA} is the use of macro moves, to make
sure that a box pushed into a corridor must be pushed the whole way through
(unless its goal is inside the corridor). We did not have time to implement
this fully, but we find the corridors, and use this to discourage multiple agents
going into the same corridor (see heuristics).

Macro moves are especially important in sokoban where it is possible to have
one way tunnels.\\

We where considering using a macro move to force an agent to use the shortest
path to get to a box, or to push/pull the box to a goal, and then only use actual
planning in case of collisions. This was abandoned since it would be difficult
to solve conflicts on such paths.\\

They use a similar definition of rooms as we do, and further describes articulation
cells which will divide a level in two parts if replaced by a wall. This is
especially important in sokoban since we cannot always move blocks, but may
also be used in our problem to store boxes in the right side of a level.\\

In \cite{highlevelSokoban} a discussion of using hierarchical search for
planing in sokoban is given. They use a highlevel planer to move boxes between
rooms by creating a graph of rooms and corridors. Inside rooms a normal lowlevel
planner is used. In our problem macro move can be used inside tunnels.

\section{Experiments/Results}
The results were calculated on a Thinkpad T420, i5 3.2 GHz, 8GB memory.
Running Arch linux, 4.5.1-1, with few simultaneous programs running and a very
lightweight window manager (no desktop environment).

Botbot was informally tested on the handout levels. We did not keep time/action
information for the different versions of botbot. As described in the discussion,
the communication broke for some multiagent levels that where previously solvable.

From our experiments we could see that storage tasks and same road heuristics enabled
us to solve some levels that we couldn't before, including the lower left part of skynet
with goals in a corridor that will block access to the last goal, though it did not
nessesarily improve on solutions that where already solveable.

The stats for the competition levels that we solved can be seen below:\\

\begin{tabular}{l|l|l}
Level & action score & time score \\
\hline
MATheAgency.lvl & 41 & 0.048 s\\
\hline
SASojourner.lvl & 250 & 0.237 s\\
SAbotbot.lvl & 115 & 0.361 s\\
SATheRedDot.lvl & 107 & 0.016 s\\
SANoOp.lvl & 722 &11.417 s\\
SAFortyTwo.lvl & 470 & 1.09 s\
\end{tabular}\\

botbot's placement in the contest can be seen below:\\

\begin{tabular}{l|l|l}
 & Number of Actions & Solution Time \\
\hline
Multiagent & 12 & 6 \\
Singleagent & 12 & 5 \\
\end{tabular}\\

These results reflects we focused on creating a fast planner instead of finding
a near optimal solution. In the multiagent track we got a decent placement, even
though we only solved one level, because we got the fastest solution time for that level.

It was not our own level, so the speed was not obtained by tuning the AI for that
specific level.

\section{Discussion}
Botbot only solved one multi agent level, and not even our own. Even though our
implementation uses APSP as heuristic, and therefore should be able to solve
levels without conflicts easily, and indeed was, at some point the multiagent
planner stopped working properly. We got problems solving a level such as MAmultiagentSort.lvl
from the test suite, which had been solved easily before, and we could not
figure out what had caused this deficiency.\\

While the communication approach of botbot should work fine when there are
no/little conflicts, it turned out to give problems when a box blocks the way for
an agent in a corridor. The agent moves away as it should, but then the fastest
way to get to the goal is to move back. If it was only a single agent, this would
mean that the state had been visited and could be discarded. For a multiagent system
however, the other agents would have moved and therefore the state is not considered
visited by A*. This causes an agent to move back and forth between two cells, and
was a real problem in MApackman.lvl.

This lead us to believe that it would be better to create full plans for each agent
individually without being aware of other and then merge them. We did not have time
to experiment with this.\\

%Summing up the heuristic makes it difficult to control. Examples of this is the
%\texttt{sameRoad} punishment. Instead of marking a move

Our use of corridors are currently very limited. We punish each time multiple agents
are in the same corridor, so that they will not go in the way of each other. This
heuristic can be much more sophisticated. For example, there are no need to punish
when multiple agents are moving though a corridor in the same direction.

As soon an agent has pushed/pulled into a corridor not containing its goal, there
is no need to let them move back and forth. These actions could be disabled to
generate less states without making the planner less general.\\

Another way to use the roads would be to put the agent on the right side of
a goal when a task is finished. This could be done by checking if the current task
has its goal inside a corridor. If so, when the agent goes inside that corridor
it must either push or pull (depending on what side of the goal is should be on).

This could also be implemented by punishing if the agent pushes where a pull is
preferred or the other way around. Then it is possible to do both push and pull
if needed, while the estimated better solution is preferred.

A problem with APSP is the lack of longer alternative routes. There are cases
where some boxes on the shortest path is expensive to move. For example if
the color is wrong, and another agent must be used. In this case it would be
effective to be able to use longer paths, perhaps going through other rooms.

Only using APSP for pathfinding heuristic makes it very hard to find solutions
that differ from the shortest path. The solutions that we considered (but did not
have time to implement) is to create waypoints using the room/corridor graph
that is already calculated. Another way would be to calculate the APSP multiple
times, considering boxes. This was abandoned since there where levels (such as bispebjerg)
where the APSP calculation took significantly time. This could possibly be countered
by only calculating the APSP rarely, like when a goal has been completed, or
after some number of actions. Further the APSP could be lazily evaluated thus
only calculating the cells that are needed.

In \cite{sokobanMA} it is suggested to use room to room transitions for
a highlevel search.\\

Goal priority never change in our implementation.\\

A number of our problems might stem from our method of development.
Though we did put some effort into a logging system to get around how the server
environment called our program, we did not have a proper testing/evaluation
framework.
When working with a new idea for a heuristic we took a level exhibiting a
problem, and tweaked our heuristics to handle this better.
Unfortunately by not rigorously evaluating the improvement of each change on
multiple/every level we were bound to hit some local optimum. This is especially
troublesome when deciding how to weigh contributions from multiple heuristics.
Thus by improving some areas we had regressions, as an example, our final
solution were no longer able to solve our own level, though it previously did it
flawlessly in little to no time. With a better testing setup we could be notified of how
each change altered the behaviour thus enabling much more focused work.

\section{Future work}
Currently the merger prioritizes the bot with lowest number. This might be ineffective,
and since we have prioritized goals, it would be obvious to use these when merging.

We would also like to calculate the individual plans individually, and merge
the complete plan, instead of merging the actions one by one. This means that
one agent going back and forth will be considered alternative states, unlike now,
where other agents changes the state when one agent does the same actions.

Another solution is to only consider part of the state when determining if it has
been visited before. Something similar was suggested in \cite{sokobanMA}, but is difficult to implement
properly.\\

We would also like to experiment with making the planner online. This would make
it unnecessary to have a negative weighted modifier to count the storage. Since
the focus is on finding the solution fast and not finding the shortest solution
using online planning is likely a good choice.\\

We where considering to use a storage value of 3 to cells that is not adjacent to
a wall, so that we can keep the walls free if there is enough storage space.
This will make the bots able to walk from room to room.\\
\section{Conclusion}
Botbot did not solve many levels. Especially the multiagent part was not
very effective. The planner did better in the time tracks compared
to the number of actions in the competition, which was our focus.

Overall the solution is not that effective. One problem is that we end up making
a lot of calculations again and again for what is almost the same state. We
might find out that one strategy is bad and leads to a deadend, but we don't
remember this when calculating a heuristic score again for another state that is
very similar but not quite the same. This overhead of recalculating the same
again and again incurs an overhead making the solution bad for a lot of levels.

Further the lack of a proper testing/evaluation framework makes it very hard to
tweak heuristics and weigh them, making it very easy to spend a lot of effort
for very little (or even negative) gain in many levels.

Our solution gave a lot of insights for how to implement various highlevel
heuristics, but unfortunately a lot of problems can't be easily deduced using a
correct guess. The architecture is most likely a deadend if more advanced levels
are the goal. But for smaller levels the approach has the potential to quicly
find a solution, even if it is not the most effective solution in number of
performed steps.

If we had to do it all over again, our first approach might actually have been
better, especially if we had come up with a solution for using our solution when
needed. Thus we could better take a highlevel approach, but when some part of
the problem cant clearly be divided into sequences of tasks, it might choose our
heuristic driven search for a smaller part of the problem. This way we would
have the best of both worlds. Of course, having such a hybrid solution comes
with its own problems, namely how to detect that the current strategy is not
optimal and to switch, and secondly how to integrate the two approaches cleanly
to ensure a valid solution is found and that the entire decision tree is searched eventually.


\bibliographystyle{aaai}
\bibliography{bibliography}
\end{document}
